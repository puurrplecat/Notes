\documentclass[12pt]{book}
\usepackage{amsmath}
\usepackage{amsthm}
\usepackage{geometry}
\geometry{
a4paper,
total={170mm,257mm},
left=20mm,
top=20mm,
}
\usepackage[upint, noamssymbols, varg]{newpxmath}
\usepackage{newpxtext}
\usepackage{enumitem}
\usepackage{xcolor}
\usepackage{graphicx}
\graphicspath{ {./images/} }
\usepackage{mdframed}
\newenvironment{xbox}
  {\vspace{1em}\begin{changemargin}{2cm}{2cm}\begin{mdframed}[linewidth=0.75pt]}
  {\end{mdframed}\end{changemargin}}
\newtheorem{theo}{Theorem}[section]
\newtheorem{prop}[theo]{Proposition}
\newtheorem{corl}[theo]{Corollary}
\newtheorem{lema}[theo]{Lemma}
\theoremstyle{definition}
\newtheorem{defi}[theo]{Definition}
\newtheorem{eggs}[theo]{Example}

\newcommand{\bb}[1]{\mathbb{#1}}

\title{Notes for the Australian Mathematics Advanced Stage 6 Course}
\date{}

\def\changemargin#1#2{\list{}{\rightmargin#2\leftmargin#1}\item[]}
\let\endchangemargin=\endlist 

\begin{document}
	\maketitle
	\tableofcontents
\part{Year 11}
\chapter{Functions}
\section{Functions and Relations}
  A function takes in input and returns exactly one output, whereas a relation can return more than one. A function is a
  mapping between two sets.
\subsection{Definition of a function}
  Functions can be thought of as a set of ordered pairs $(x,y)$.
\subsection{Domain, range, independent and dependent variables}
  \begin{defi}[Domain and range]
  	The domain of a function is the set of all possible inputs values a function can have. The range of a function is 
  	the set of all possible outputs a function can have.
  \end{defi}
  \begin{defi}[Independent and dependent variables]
  	A function $f(x)$ has $x$ as the independent variable. The output (or $f(x)$ itself) is the dependent variable, as it
  	is dependent on $x$.
  \end{defi}
  \noindent Interval notation is used to describe unbroken portions of the real line.
  \begin{defi}[Interval notation]
  	The \textit{open} interval $(a,b)$ represents the set of all real numbers between $a$ and $b$ but not including $a$ or $b$.
  	The \textit{closed} interval $[a,b]$ contains all real numbers between $a$ and $b$ and includes the endpoints $a$ and $b$. 
  	The \textit{half-open} interval $[a,b)$ is the set of all real numbers between $a$ and $b$ but only includes the endpoint $a$.
  	Similarly, the half-open interval $(a,b]$ only contains the endpoint $b$.
  \end{defi}
  Interval notation is used with the $\in$ symbol (read as ``in'') You may use the notation $x\in\bb{R}$ to denote $x$ being any 
  real number. Intervals are combined with the union (meaning ``or'') symbol, such as in \[x\in(-\infty,0)\cup(0,\infty)\]
\subsection{Graphs of functions}
\subsection{Types of functions and relations}
  \begin{defi}[One-to-one]
    A \textit{one-to-one} function takes in one input and returns exactly one output. It passes both
    horizontal and vertical line tests. For example, \[f(x)=x\] is a one-to-one function.
  \end{defi}
  \begin{defi}[Many-to-one]
    A \textit{One-to-many} function has more than one input that produce the same output. It fails
    the horizontal line test but passes ther vertical. An example is \[f(x)=x^2\] where both $x=-1$ and $x=1$ give the
    output $1$.
  \end{defi}
  \begin{defi}[One-to-many]
    A \textit{one-to-many} relation has an input that outputs more than one number. It passes the horizontal line test but
    fails the vertical. An example relation is \[y^2=x\] where $x=1$ has the outputs $-1$ and $+1$. 
  \end{defi}
  \begin{defi}[Many-to-many]
    A \textit{many-to-many} relation has multiple input which output multiple of the same outputs. It fails both horizontal 
    and vertical line tests. An example is the circle \[x^2+y^2=1\] where $x=1$ or $x=-1$ both produce outputs $1$ and $-1$.
  \end{defi}
\subsection{Properties of functions}
  \begin{defi}[Even function]
    A function $f$ is even if it satisfies, \[f(-x)=f(x).\] The graph of $f$ is symmetric about
    the $y$-axis.
  \end{defi}
  \begin{defi}[Odd function]
    A function $f$ is odd if it satisfies, \[f(-x)=-f(x).\] The graph of $f$ has point symmetry about the origin (if you
    spin the graph $180^\circ$ about the origin, its the same graph).
  \end{defi}
  \begin{defi}[Algebra of functions]
    Two functions $f$ and $g$ can be added ($f(x)+g(x)$), subtracted ($f(x)-g(x)$), multiplied ($f(x)g(x))$ together or 
    divided ($f(x)/g(x)$ provided $g(x)$ is never $0$), forming a new function. The domain of the new function is the intersection of the domain of $f$
    and the domain of $g$. The range is more difficult to find.
  \end{defi}
  \begin{defi}[Function composition]
    Function composition is another way to combine functions to form new functions. The composition of functions $f$ and $g$
    is denoted as \[f\circ g(x)=f(g(x)).\] The domain of $f\circ g$ is the domain of $g$, whose outputs must also lie in the
    domain of $f$. The range of $f\circ g$ is all the outputs that from the range of $g$ as input.
  \end{defi}
\subsection{Solutions to functions}
  When we solve the equation \[f(x)=0,\] we are solving for the $x$ values that are sent to $0$ by the function. Graphically,
  they are the $x$-intercepts. This is because we are finding all ordered pairs whose $y$ (output) value is $0$, corresponding
  to the $x$-intercepts.
\section{Linear, quadratic and cubic functions}
\subsection{Linear functions}
  A linear function is of the form \[f(x)=mx+c\] where $m$ and $c$ are any real numbers. $m$ represents the gradient and
  $c$ is the $y$-intercept. This produces a straight line graph.
  \begin{defi}[Point-gradient form]
    The unique line that passes through the point $(x_1,y_1)$ and has a gradient of $m$ has the equation
    \[y-y_1=m(x-x_1)\] 
  \end{defi}
  If you had only two points and no gradient, first calculate the gradient from the two points, and use the above equation.
  The equation for gradient between two points $(x_1,y_1)$ and $(x_2,y_2)$ is \[m=\frac{y_2-y_1}{x_2-x_1}.\]
  \begin{defi}[Parallel and perpendicular lines]
    Two lines a parallel if and only if their gradients are equal. If their $y$-intercepts are also equal then they are they
    exact same line. Two lines are perpendicular if and only if their gradients are negative recipricols of each other. That
    is, if two gradients are $m_1$ and $m_2$, then \[m_1m_2=-1.\]
  \end{defi}
\subsection{Quadratic functions}
  A quadratic function is of the form \[f(x)=ax^2+bx+c\] for some real numbers $a$, $b$ and $c$.
  \begin{defi}[Vertex]
  The \textit{vertex} (or \textit{turning point}) of a parabola is where the curve of the parabola changes direction. The $x$-value
  of the vertex is \[x=\frac{-b}{2a}.\] The parabola is symmetric across the line containing the vertex.
  \end{defi}
  Parabolas may have zero, one or two vertices, which also means the equation $f(x)=0$ has zero, one or two solutions.
  \begin{defi}[Completing the square]
    Given a parabola of the form \[f(x)=ax^2+bx+c,\] completing the square allows us to form a square term plus a constant.
  \end{defi}
  Steps to completing the square:
  \begin{enumerate}[label=\roman*)]
    \item Factor out the coefficient of $x^2$. \[f(x)=a\left(x^2+\frac{b}{a}x+\frac{c}{a}\right)\]
    \item Add and subtract the square of the half of the coefficient of $x$ in the bracket. 
      \[f(x)=a\left(x^2+\frac{b}{a}x+\frac{b^2}{4a^2}+\frac{c}{a}-\frac{b^2}{4a^2}\right)\]
    \item The first three terms above form a perfect square \[f(x)=a\left(\left(x-\frac{b}{2a}\right)^2+\frac{c}{a}-\frac{b^2}{4a^2}\right)\]
    \item Expand the brackets.
      \[f(x)=a\left(x-\frac{b}{2a}\right)^2+c-\frac{b^2}{4a}\]
  \end{enumerate}
  We also call this form \textit{vertex form}. If \[f(x)=a(x-h)^2+k\] then the vertex is at $(h,k)$
  \bigskip\\ Factorising quadratics may be done in many ways. I propose only one.
  \begin{eggs}[Factorising quadratics]
    Suppose we have to factorise \[f(x)=6x^2-17x+12.\] We need to find two numbers $a$ and $b$ that 
    \begin{itemize}
      \item MULTIPLY to $72$,
      \item SUM to $-17$.
    \end{itemize}
    You start listing pairs of positive numbers that multiple to $72$ in your head (or on paper). Whilst you scan for pairs, think about
    which pair could possibly combined (with addition and subtraction) to make $-17$. That pair is $-9$ and $-8$. Check that
    the sign of the product is positive. Split the middle term into $-9x$ and $-8x$ to get
    \[f(x)=(6x^2-9x)+(-8x+12),\]
    group the terms as such and factorise.
  \end{eggs}
  \begin{defi}[The quadratic formula]
    Given a quadratic \[f(x)=ax^2+bx+c,\] the solutions to $f(x)=0$ are
    \[x=\frac{-b\pm\sqrt{b^2-4ac}}{2a}.\] We call $b^2-4ac$ the \textit{discriminant} of the parabola.
  \end{defi}
  \begin{defi}[Discriminant]
    The discriminant of a quadratic function $f(x)=ax^2+bx+c$ is \[\Delta=b^2-4ac.\]
    \begin{enumerate}
      \item If $\Delta>0$, there are two solutions. The graph does not touch the $x$-axis.
      \item If $\Delta=0$, there is one solution. The vertex is the only point that touches the $x$-axis.
      \item If $\Delta<0$, there is no solutions. The graph passes through the $x$-axis at different points.
    \end{enumerate}
  \end{defi}
\subsection{Simultaneous equations}
\part{Year 12}
\end{document}
