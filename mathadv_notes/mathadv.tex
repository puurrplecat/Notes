\documentclass[12pt]{book}
\usepackage{amsmath}
\usepackage{amsthm}
\usepackage{geometry}
\geometry{
a4paper,
total={170mm,257mm},
left=20mm,
top=20mm,
}
\usepackage[upint, noamssymbols]{newtxmath}
\usepackage{newpxtext}
\usepackage{enumitem}
\usepackage{xcolor}
\usepackage{graphicx}
\graphicspath{ {./images/} }
\usepackage{mdframed}
\usepackage{tabularx}
\newenvironment{xbox}
  {\vspace{1em}\begin{changemargin}{2cm}{2cm}\begin{mdframed}[linewidth=0.75pt]}
  {\end{mdframed}\end{changemargin}}

\DeclareMathOperator{\var}{Var}
  
\newtheorem{theo}{Theorem}[section]
\newtheorem{prop}[theo]{Proposition}
\newtheorem{corl}[theo]{Corollary}
\newtheorem{lema}[theo]{Lemma}
\theoremstyle{definition}
\newtheorem{defi}[theo]{Definition}
\newtheorem{eggs}[theo]{Example}

\newcommand{\bb}[1]{\mathbb{#1}}

\title{Notes for the Australian Mathematics Advanced Stage 6 Course}
\date{}

\def\changemargin#1#2{\list{}{\rightmargin#2\leftmargin#1}\item[]}
\let\endchangemargin=\endlist 

\begin{document}
	\maketitle
	\tableofcontents
\part{Year 11}
\chapter{Functions}
\section{Functions and Relations}
  A function takes in input and returns exactly one output, whereas a relation can return more than one. A function is a
  mapping between two sets.
\subsection{Definition of a function}
  Functions can be thought of as a set of ordered pairs $(x,y)$.
\subsection{Domain, range, independent and dependent variables}
  \begin{defi}[Domain and range]
  	The domain of a function is the set of all possible inputs values a function can have. The range of a function is 
  	the set of all possible outputs a function can have.
  \end{defi}
  \begin{defi}[Independent and dependent variables]
  	A function $f(x)$ has $x$ as the independent variable. The output (or $f(x)$ itself) is the dependent variable, as it
  	is dependent on $x$.
  \end{defi}
  \noindent Interval notation is used to describe unbroken portions of the real line.
  \begin{defi}[Interval notation]
  	The \textit{open} interval $(a,b)$ represents the set of all real numbers between $a$ and $b$ but not including $a$ or $b$.
  	The \textit{closed} interval $[a,b]$ contains all real numbers between $a$ and $b$ and includes the endpoints $a$ and $b$. 
  	The \textit{half-open} interval $[a,b)$ is the set of all real numbers between $a$ and $b$ but only includes the endpoint $a$.
  	Similarly, the half-open interval $(a,b]$ only contains the endpoint $b$.
  \end{defi}
  Interval notation is used with the $\in$ symbol (read as ``in'') You may use the notation $x\in\bb{R}$ to denote $x$ being any 
  real number. Intervals are combined with the union (meaning ``or'') symbol, such as in \[x\in(-\infty,0)\cup(0,\infty)\]
\subsection{Graphs of functions}
\subsection{Types of functions and relations}
  \begin{defi}[One-to-one]
    A \textit{one-to-one} function takes in one input and returns exactly one output. It passes both
    horizontal and vertical line tests. For example, \[f(x)=x\] is a one-to-one function.
  \end{defi}
  \begin{defi}[Many-to-one]
    A \textit{One-to-many} function has more than one input that produce the same output. It fails
    the horizontal line test but passes ther vertical. An example is \[f(x)=x^2\] where both $x=-1$ and $x=1$ give the
    output $1$.
  \end{defi}
  \begin{defi}[One-to-many]
    A \textit{one-to-many} relation has an input that outputs more than one number. It passes the horizontal line test but
    fails the vertical. An example relation is \[y^2=x\] where $x=1$ has the outputs $-1$ and $+1$. 
  \end{defi}
  \begin{defi}[Many-to-many]
    A \textit{many-to-many} relation has multiple input which output multiple of the same outputs. It fails both horizontal 
    and vertical line tests. An example is the circle \[x^2+y^2=1\] where $x=1$ or $x=-1$ both produce outputs $1$ and $-1$.
  \end{defi}
\subsection{Properties of functions}
  \begin{defi}[Even function]
    A function $f$ is even if it satisfies, \[f(-x)=f(x).\] The graph of $f$ is symmetric about
    the $y$-axis.
  \end{defi}
  \begin{defi}[Odd function]
    A function $f$ is odd if it satisfies, \[f(-x)=-f(x).\] The graph of $f$ has point symmetry about the origin (if you
    spin the graph $180^\circ$ about the origin, its the same graph).
  \end{defi}
  \begin{defi}[Algebra of functions]
    Two functions $f$ and $g$ can be added ($f(x)+g(x)$), subtracted ($f(x)-g(x)$), multiplied ($f(x)g(x))$ together or 
    divided ($f(x)/g(x)$ provided $g(x)$ is never $0$), forming a new function. The domain of the new function is the intersection of the domain of $f$
    and the domain of $g$. The range is more difficult to find.
  \end{defi}
  \begin{defi}[Function composition]
    Function composition is another way to combine functions to form new functions. The composition of functions $f$ and $g$
    is denoted as \[f\circ g(x)=f(g(x)).\] The domain of $f\circ g$ is the domain of $g$, whose outputs must also lie in the
    domain of $f$. The range of $f\circ g$ is all the outputs that from the range of $g$ as input.
  \end{defi}
\subsection{Solutions to functions}
  When we solve the equation \[f(x)=0,\] we are solving for the $x$ values that are sent to $0$ by the function. Graphically,
  they are the $x$-intercepts. This is because we are finding all ordered pairs whose $y$ (output) value is $0$, corresponding
  to the $x$-intercepts.
\section{Linear, quadratic and cubic functions}
\subsection{Linear functions}
  A linear function is of the form \[f(x)=mx+c\] where $m$ and $c$ are any real numbers. $m$ represents the gradient and
  $c$ is the $y$-intercept. This produces a straight line graph.
  \begin{defi}[Point-gradient form]
    The unique line that passes through the point $(x_1,y_1)$ and has a gradient of $m$ has the equation
    \[y-y_1=m(x-x_1)\] 
  \end{defi}
  If you had only two points and no gradient, first calculate the gradient from the two points, and use the above equation.
  The equation for gradient between two points $(x_1,y_1)$ and $(x_2,y_2)$ is \[m=\frac{y_2-y_1}{x_2-x_1}.\]
  \begin{defi}[Parallel and perpendicular lines]
    Two lines a parallel if and only if their gradients are equal. If their $y$-intercepts are also equal then they are they
    exact same line. Two lines are perpendicular if and only if their gradients are negative recipricols of each other. That
    is, if two gradients are $m_1$ and $m_2$, then \[m_1m_2=-1.\]
  \end{defi}
\subsection{Quadratic functions}
  A quadratic function is of the form \[f(x)=ax^2+bx+c\] for some real numbers $a$, $b$ and $c$.
  \begin{defi}[Vertex]
  The \textit{vertex} (or \textit{turning point}) of a parabola is where the curve of the parabola changes direction. The $x$-value
  of the vertex is \[x=\frac{-b}{2a}.\] The parabola is symmetric across the line containing the vertex.
  \end{defi}
  Parabolas may have zero, one or two vertices, which also means the equation $f(x)=0$ has zero, one or two solutions.
  \begin{defi}[Completing the square]
    Given a parabola of the form \[f(x)=ax^2+bx+c,\] completing the square allows us to form a square term plus a constant.
  \end{defi}
  Steps to completing the square:
  \begin{enumerate}[label=\roman*)]
    \item Factor out the coefficient of $x^2$. \[f(x)=a\left(x^2+\frac{b}{a}x+\frac{c}{a}\right)\]
    \item Add and subtract the square of the half of the coefficient of $x$ in the bracket. 
      \[f(x)=a\left(x^2+\frac{b}{a}x+\frac{b^2}{4a^2}+\frac{c}{a}-\frac{b^2}{4a^2}\right)\]
    \item The first three terms above form a perfect square \[f(x)=a\left(\left(x-\frac{b}{2a}\right)^2+\frac{c}{a}-\frac{b^2}{4a^2}\right)\]
    \item Expand the brackets.
      \[f(x)=a\left(x-\frac{b}{2a}\right)^2+c-\frac{b^2}{4a}\]
  \end{enumerate}
  We also call this form \textit{vertex form}. If \[f(x)=a(x-h)^2+k\] then the vertex is at $(h,k)$
  \bigskip\\ Factorising quadratics may be done in many ways. I propose only one.
  \begin{eggs}[Factorising quadratics]
    Suppose we have to factorise \[f(x)=6x^2-17x+12.\] We need to find two numbers $a$ and $b$ that 
    \begin{itemize}
      \item MULTIPLY to $72$,
      \item SUM to $-17$.
    \end{itemize}
    You start listing pairs of positive numbers that multiple to $72$ in your head (or on paper). Whilst you scan for pairs, think about
    which pair could possibly combined (with addition and subtraction) to make $-17$. That pair is $-9$ and $-8$. Check that
    the sign of the product is positive. Split the middle term into $-9x$ and $-8x$ to get
    \[f(x)=(6x^2-9x)+(-8x+12),\]
    group the terms as such and factorise.
  \end{eggs}
  \begin{defi}[The quadratic formula]
    Given a quadratic \[f(x)=ax^2+bx+c,\] the solutions to $f(x)=0$ are
    \[x=\frac{-b\pm\sqrt{b^2-4ac}}{2a}.\] We call $b^2-4ac$ the \textit{discriminant} of the parabola.
  \end{defi}
  \begin{defi}[Discriminant]
    The discriminant of a quadratic function $f(x)=ax^2+bx+c$ is \[\Delta=b^2-4ac.\]
    \begin{enumerate}
      \item If $\Delta>0$, there are two solutions. The graph does not touch the $x$-axis.
      \item If $\Delta=0$, there is one solution. The vertex is the only point that touches the $x$-axis.
      \item If $\Delta<0$, there is no solutions. The graph passes through the $x$-axis at different points.
    \end{enumerate}
  \end{defi}
\subsection{Simultaneous equations}
\subsection{The equation $f(x)=k$}
  When we algebraically solve the equation $f(x)=k$, graphically, we are finding the point of intersection of the graph
  of $f$ with the horizontal line $y=k$.
\subsection{Application: break-even point}
\subsection{Cubic functions}
  The cubic functions we will study here are of the form
  \begin{align}
  f(x) &=kx^3\\
    \text{or}\ f(x) &=k(x-b)^3+c\\
    \text{or}\ f(x) &=k(x-a)(x-b)(x-c)
  \end{align}
  where $a,b,c,k$ are real numbers.
  \begin{itemize}
    \item For Function 1.1, its only turning point is at $(0,0)$ and increasing $|k|$ makes it more narrow. The sign of $k$
    determines the sign of the function as $x\to\pm\infty$.
    \item For Function 1.2, its only turning point is at $(b,c)$, and is only a shifted Function 1.1.
    \item For Function 1.3, it has three $x$-intercepts at $a$, $b$ and $c$. The sign of $k$ determines where the function
    starts and ends.
  \end{itemize}
\section{Further functions and relations}
\subsection{Polynomials and the polynomial functions}
  \begin{defi}[Polynomials and polynomial function]
    A polynomial is an expression of the form \[a_nx^n+a_{n-1}x^{n-1}+\cdots+a_1x+a_0\]
    where $n$ is a nonnegative ($\geq 0$) integer, $a_1,\ldots,a_n$ are real numbers with $a_n\neq 0$. A polynomial
    function is a function built from the polynomial expression, that is
    \[p(x)=a_nx^n+a_{n-1}x^{n-1}+\cdots+a_1x+a_0\]
  \end{defi}
  \begin{defi}[Polynomial terms]
     Let $p$ be a polynomial function, that is
    \[p(x)=a_nx^n+a_{n-1}x^{n-1}+\cdots+a_1x+a_0.\]
    We call $n$ the \textit{degree} of the polynomial. The coefficient $a_n$ is called the \textit{leading coefficient}.
  \end{defi}
  You will often come across polynomials in their factored form, and asked to sketch. That is, to sketch something of the form
  \[p(x)=k(x-\alpha_1)(x-\alpha_2)\cdots(x-\alpha_n).\]
  First, all the roots are also the $x$-intercepts. Then find the sign of the leading coefficient (sign of $k$ above) as that tells
  you where the function starts and ends.
\subsection{The hyperbolic function}
  The hyperbolic function is important to study because they represent quantities that decrease rapidly as its dependent variable
  increases.
  \begin{defi}[The hyperbolic function]
    The hyperbolic function we will study here is of the form \[f(x)=\frac{k}{x}.\] It has asymptotes $x=0$ and $y=0$.
  \end{defi}
\subsection{The absolute value function}
  The use of the absolute value function is to find the ``size'' of real numbers. That is the size of $-2$ is $2$ because it
  is a distance of $2$ away from $0$ on the numberline.
  \begin{defi}[The absolute value function]
    The basic absolute value function is \[f(x)=|x|\] and is defined by the piecewise function
    \[f(x)=\begin{cases}x &\text{if}\ x\geq 0\\-x &\text{if}\ x<0\end{cases}.\]
    We can generalise the above function further to absolute functions of the form
    \[f(x)=|ax+b|\]
    where $a$ and $b$ are real numbers. The piecewise function that defines these are
    \[f(x)=\begin{cases}ax+b &\text{if}\ x\geq \frac{-b}{a}\\ \\ -ax-b &\text{if}\ x<\frac{-b}{a}\end{cases}.\]
  \end{defi}
  \noindent How to sketch $f(x)=|ax+b|$.
  \begin{enumerate}
    \item Find the $x$ intercept by solving $ax+b=0$. Then sketch each seperate line according to the cases above.
  \end{enumerate}
  How to solve absolute value equations. Let us try to solve $|ax+b|=k$.
  \begin{enumerate}
    \item Split the equation into two equations. That is
      \[\begin{cases}ax+b=k &\text{if}\ x\geq \frac{-b}{a}\\ \\ -ax-b=k &\text{if}\ x<\frac{-b}{a}\end{cases}.\]
    \item Solve each case seperately. Check that your solution for $x$ lies in the corresponding restriction $x$.
  \end{enumerate}
  Graphically, we are finding the point of intersection between $|ax+b|$ and $y=k$.
\subsection{Function transormations}
  \begin{theo}
    Given a function $f(x)$, the function $-f(x)$ is the function $f$ reflected across the $x$-axis.
  \end{theo}
  \begin{theo}
    Given a function $f(x)$, the function $f(-x)$ is the function $f$ reflected across the $y$-axis.
  \end{theo}
  \begin{theo}
    Given a function $f(x)$, the function $-f(-x)$ is the function $f$ reflected across the $x$-axis, then reflected
    across the $y$ axis.
  \end{theo}
\subsection{Circles}
  A circle with radius $r$ and centre at the origin has the equation \[x^2+y^2=r^2.\] Note this is not a function.
  A general circle with radius $r$ and centre at $(a,b)$ has the equation \[(x-a)^2+(y-b)^2=r^2.\]
  The equation is derived with the Pythagoras' theorem. If the question gave $x^2+y^2+ax+by+c=0$, complete the square
  to get the equation into the form $(x-a)^2+(y-b)^2=r^2$.
\subsection{Semicircles}
  A positive semicircle with radius $r$ and centre at the origin has the equation \[y=\sqrt{r^2-x^2}.\] A negative semicircle
  has the equation \[y=-\sqrt{r^2-x^2}.\] Both of these semicircles are functions.


\chapter{Trigonometric Functions}
\section{Trigonometry and measure of angles}
\subsection{Introduction to trigonometry}
  \begin{align*}
    \sin\theta &=\frac{\text{opposite}}{\text{hypotenuse}}\\
    \cos\theta &= \frac{\text{adjacent}}{\text{hypotenuse}}\\
    \tan\theta &= \frac{\text{opposite}}{\text{adjacent}}
  \end{align*}
  This can be remembered with the handy mnemonic SOH CAH TOA. Pronounced (SO-KAH-TOWA).
\subsection{Special formulae}
  \begin{theo}[Sine rule]
    Let a triangle have angles $A$, $B$, $C$, and side lengths $a$, $b$, $c$ opposite their respective angles. The sine rule
    states that
    \[\frac{\sin A}{a}=\frac{\sin B}{b} = \frac{\sin C}{c}.\]
    Consequently, we have that \[\frac{a}{\sin A}=\frac{b}{\sin B}=\frac{c}{\sin C}\]
  \end{theo}
  \begin{theo}[Cosine rule]
    Let a triangle have angles $A$, $B$, $C$, and side lengths $a$, $b$, $c$ opposite their respective angles. The cosine rule
    states that
    \[c^2=a^2+b^2-2ab\cos C\]
    Similarly, $a^2 = b^2+c^2-2bc\cos A$ and $b^2=a^2+c^2-2ac\cos B$. There are also the formulas for the angle at a vertex
    \[C=\cos^{-1}\left(\frac{a^2+b^2-c^2}{2ab}\right).\] The angle formulas are similar for the other angles $B$ and $A$.
  \end{theo}
  \begin{theo}[Area of a triangle]
    Let a triangle have two sides $a$ and $b$, and an included angle (angle between the sides) $C$. Then the area of the triangle
    is given by \[A=\frac{1}{2}ab\sin C.\]
  \end{theo}
  DON'T FORGET THE AMBIGUOUS CASE.
  \begin{defi}[Angle of elevation and depression]
    The angle of elevation an object makes with an observer, is the angle the object makes with the horizontal. The observer
    has to ``look'' up for it to be called an angle of elevation. Similarly, the angle of elevation is the angle the object makes
    with the horizontal, but the observer has to ``look'' down.
  \end{defi}
\chapter{Calculus}
\chapter{Exponential and logarithmic functions}
\chapter{Statistical analysis}
\section{Probability}
  Experimental probability is measuring probability from an actual experiment conducted in reality, whereas theoretical probability
  is a mathematically calculated expectation of outcomes. The probability calculated from experiments is called \textit{relative
  frequency} and is calculated with the formula
  \[\frac{\text{favourable outcomes}}{\text{total outcomes}},\] where the number of outcomes occured in some scenario. Probabilities
  can be visualised on a \textit{probability scale}, which is a line from 0 to 1.
\subsubsection*{Strategies}
  \begin{tabularx}{\linewidth}{l X}
  Exhaustive listing: & You list every possible outcome, and count all the favourable outcomes. \\ \\
  Arrays (two stages): & You draw a table. The top row and left column represent the outcomes of two events. The other boxes
    represent the all the combinations of the two events. Count the favourable outcomes. \\ \\ 
  Tree (multi-stage): & Each event has certain outcomes, which form branches of a tree. A full branch represents a final outcome.  
  \end{tabularx}
\subsection{Set notation}
  We can describe events by encapsulating them in sets. Sets are denoted with capital letters such as $A$ and this will represent
  an event like shoe sizes.
  \begin{defi}[Complement]
    The complement of a set $A$ is denoted with $\overline{A}$ (or $A'$ or $A^c$) and represents all outcomes that aren't in $A$.
    For example, if $A$ represented even numbers, then $\overline{A}$ would be the odd numbers.
  \end{defi}
  \begin{defi}[Union]
    The \textit{union} of two sets $A$ and $B$ is denoted $A\cup B$ and is the collection of all events in $A$ or in $B$.
  \end{defi}
  \begin{defi}[Intersection]
    The \textit{intersection} of two sets $A$ and $B$ is denoted $A\cap B$ and is the collection of all events that occur
    in both $A$ and $B$.
  \end{defi}
  Drawing and shading Venn diagrams are very helpful.
  \begin{prop}[Some rules]
    \ \begin{enumerate}
      \item $P(\overline{A})=1-P(A)$
      \item $P(A\cup B)=P(A)+P(B)-P(A\cap B)$
    \end{enumerate}
  \end{prop}
\subsection{Independence and mutually exclusive}
  Events are mutually exclusive if they cannot happen at the same time. For example, walking forwards and backwards are mutually
  exclusive events. Independent events are events whose outcomes do not affect each other.
  \vspace{1em} \\
  If two events $A$ and $B$ are indepenent, then $P(A\cap B)=P(A)\times P(B)$. (This is also known as the multiplication law.)
\subsection{Conditional probability}
  For dependent events, certain outcomes can change the probabilites of another event's outcomes, but their probability can still
  be calculated. Let's say $A$ depends on $B$, and that $B$ occurs first. We can calculate the probability of $A$ occuring \textit{given}
  that $B$ has occured. We use the formula
  \[P(A\mid B)=\frac{P(A\cap B)}{P(B)}.\]
  (Note that $P(B)\neq 0$.) If $A$ and $B$ were instead independent, then $P(A\mid B)=P(A)$.
\section{Discrete probability distributions}
  \begin{defi}[Random variable]
    A \textit{random variable} (denoted with a capital letter) represents the outcomes of a random event (Truthfully incorrect). For example, if an event rolling a dice, then the random
    variable would represent the outcomes 1, 2, 3, 4, 5 and 6. The previous example is also an example of a \textit{discrete} random
    variable. A continuous one could be human height. 
  \end{defi}
  In this course, random variables will always be numbers. There are also probabilities associated with random variables. Let's say
  our event is one dice rolled, and so our random variable $X$ will be the numbers 1 to 6. We construct a 
  discrete probability distribution table.
  \begin{center}\begin{tabular}{|c|c|c|c|c|c|c|}
  \hline &&&&&&\\[-0.6em]
  X & 1 & 2 & 3 & 4 & 5 & 6 \\[-0.6em] &&&&&& \\
  \hline
  &&&&&&\\[-0.6em]
  P(X=x) & 1/6 & 1/6 & 1/6 & 1/6 & 1/6 & 1/6\\[-0.6em] &&&&&& \\
  \hline
  \end{tabular}\end{center}
  When the probabilites are all equal (like in the above example), we call the distribution \textit{uniform}.
\subsubsection*{Conditions for probability distributions}
  \begin{enumerate}
    \item The sum of all probabilities is 1.
    \item There are no negative probabilities.
  \end{enumerate}
\subsection{Expected value}
  The expected value of a random variable $E(X)$ represents the mean outcome of an event. For example, a game requires you to roll prime on
  a ten-sided die to win \$5 otherwise it is a loss of \$3. What can you \textit{expect} to win from this game? The first step
  is to construct a probability distribution table. (What is the random variable?)
  \begin{center}\begin{tabular}{|c|c|c|}
  \hline &&\\[-0.6em]
  $X$ & $-3$ & $5$ \\[-0.6em] && \\
  \hline
  &&\\[-0.6em]
  $P(X=x)$ & $3/5$ & $2/5$ \\[-0.6em] && \\
  \hline
  \end{tabular}\end{center}
  The expected value is calculated by multiplying each of the random variable outcomes by their corresponding probability, and
  summing it all up. For this example, the working out will be
  \[-3\times\frac{3}{5}+5\times\frac{2}{5}=0.2,\] so we can conclude that we can expect to win $\$0.20$.
  \vspace{1em}\\
  Since expected value is the mean, it is sometimes written as $\mu$.
\subsection{Variance}
  Variance (denoted $\var(X)$) is a measures how spread out outcomes are (the larger the variance, the more spread out the outcomes are). It calculated 
  from the formula \[E(X^2)-\left(E(X)\right)^2\]. The value $E(X^2)$ is calculated similarly to mean, except the square of the random
  variable is multiplied to the probability before it is summed. It is useful to write a third row in the table for $X^2$.
  \begin{center}\begin{tabular}{|c|c|c|}
  \hline &&\\[-0.6em]
  $X$ & $-3$ & $5$ \\[-0.6em] && \\
  \hline
  &&\\[-0.6em]
  $X^2$ & $9$ & $25$ \\[-0.6em] && \\
  \hline && \\[-0.6em]
  $P(X=x)$ & $3/5$ & $2/5$ \\[-0.6em] && \\
  \hline
  \end{tabular}\end{center}
  Then \[\begin{aligned}E(X^2) &=9\times\frac{3}{5}+25\times\frac{2}{5}\\ &= 15.4\end{aligned}\] and since we already know
  that $E(X)=0.2$, then 
  \[\var(X)=15.4-0.2^2=15.36.\]
\subsection{Standard deviation}
  Standard deviation (denoted $\sigma$) is another measure of spread and is the square root of variance.
  \[\sigma=\sqrt{\var(X)}.\]
  It exists because it is a more useful descriptor of spread (as you will see applications of in Year 12).
\subsubsection*{Note on sample vs population}
  A \textit{population} is the complete total something, like all people or every brand. However it is unrealistic to survey or observe so many
  things. So a sample is taken where only a portion of the population is examined. This means that all statistical measurements like
  mean or variance of the sample slightly differ from the measurements of the population. (On a calculator you will use the population
  mean and variance even if your data is sampled, otherwise the calculator will use a different formula.) Obviously, as our sample
  grows larger, our measurements get closer to measurements from population.
\part{Year 12}
\end{document}
